\vspace{-1.1em}

\cvsection{Projects}

\vspace{0.2em}

\begin{cvparagraph}

\begin{itemize}[leftmargin=*]
    \itemsep-0.6em
    % \item {\textbf{High Performance Computing:} parallelised Lattice-Boltzmann codes with the OpenCL, OpenMP and MPI libraries to run on both GPUs and CPUs. All implementations were written in C. Achieved the highest grade in the class. Available on \href{https://github.com/ainsleyrutterford/HPC-OpenCL}{GitHub}.}
    % \item {\textbf{Computer Architecture:} implemented a simulation of a superscalar, out-of-order processor with register renaming and dynamic branch prediction. From scratch using Python. Available on \href{https://github.com/ainsleyrutterford/Superscalar}{GitHub}.}
    \item{\textbf{\href{https://short.as}{short.as}:} designed and implemented a free URL shortening service capable of 40 million URL shortens per day. Implemented a serverless backend using API Gateway, Lambda, and DynamoDB. The Lambdas are running TypeScript using AWS LLRT for P50 runtimes of 23 ms, and P99 of 96 ms for the URL shorten operation. Load tested up to 450 TPS with 0 errors after 100,000 URL shorten operations. The site is hosted on S3 with a CloudFront distribution using edge functions to allow a single domain to be used for both the backend API and the website. Monthly cost for 4 million daily URL shortens and 40 million daily long URL fetches would be just \$10. Frontend written in Next.js and React with TypeScript. Source code available on \href{https://github.com/ainsleyrutterford/short.as}{GitHub}.}
    % Design document and cost breakdown available \href{https://here}{here}
    \item{\textbf{\href{https://cpbitmap.github.io/}{cpbitmap.github.io}:} implemented a free browser based tool that converts Apple\textquotesingle s proprietary CPBitmap image format to PNG, JPEG, or TIFF. Uses Next.js to generate static files that are automatically deployed to GitHub pages via GitHub Actions. Frontend written in React using TypeScript. Source code available on \href{https://github.com/cpbitmap/cpbitmap.github.io}{GitHub}.}
    \item{\textbf{Deep Learning Thesis Project:} implemented and ablated the U-Net CNN architecture using Keras. It successfully extracts the density banding information present in coral skeleton CT scans, enabling researchers to more accurately calculate coral growth rates. The results were published in the SN Applied Sciences Journal (\href{https://doi.org/10.1007/s42452-021-04912-x}{DOI}). Source code available on \href{https://github.com/ainsleyrutterford/Thesis}{GitHub}.}
    % \item{\textbf{Deep Learning:} wrote a small deep learning library from scratch in Python enabling the creation, training, and inference of fully connected neural networks. Available on \href{https://github.com/ainsleyrutterford/tinyNet}{GitHub}.}
    % \item {\textbf{Cryptography:} implemented a Differential Power Analysis (DPA) attack in Python that targets software AES implementations. Wrote an AES software implementation in C that utilises various DPA countermeasures such as ‘masking’.}
    % \item {\textbf{Computer Graphics:} created a 3D rasteriser using C++ and the SDL, and GLM libraries. Implemented full triangle clipping, shadow volumes, and anti-aliasing. Also using C++, SDL, and GLM, implemented a real-time raytracer with soft shadows and anti-aliasing. Available on \href{https://github.com/ainsleyrutterford/Rasteriser}{GitHub}.}
    % \item {\textbf{Web Development:} implemented the backend and frontend of \href{https://timeline.wang}{an interactive 3D timeline} using pure HTML, CSS and JavaScript. The pseudo 3D animation uses no frameworks or libraries. Available on \href{https://github.com/ainsleyrutterford/timeline.wang}{GitHub}.}
\end{itemize}

\end{cvparagraph}
